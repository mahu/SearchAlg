\documentclass[11pt]{article}

\usepackage{microtype}
\usepackage{amsmath}
\usepackage{xcolor}
\usepackage{graphicx}
\usepackage{enumitem}


\setlength{\parindent}{0cm}
\renewcommand\thesubsection{\alph{subsection})}

\title{\textbf{Assignment 1\\}Search Algorithms}
\author{Malik Al-hallak 90020\\
		Sebastian Utzig 100059\\
		Clemens Wegener 91268}
\date{}
\begin{document}

\maketitle
\section{Computational complexity theory}

\setcounter{subsection}{1} %counter manuell erhöhen
\subsection{}
$f(n) = \Omega{(g(n))} \Longleftrightarrow \exists\: c>0,\exists \: n_0>0:c\cdot |g(n)|\leq|f(n)|$

\section{Computational complexity theory}
\subsection{}
\emph{Computational Model}: NP problems can be solved by a non-deterministic Turing machine in polynomial time.

\bigskip

\emph{Ressource Limit}: \begin{itemize}[nosep]
							\item computational time
							\item \# of steps for solution
							\item memory 
						\end{itemize}

\subsection{}
NP-Hard problems define a class of problems whose solution algorithm working in polynomial time leads to a polynomial solution algorithm for problems in NP. Vice versa all problems in NP can be reduced to NP-Hard problems.

\subsection{}
Every NP-Hard Problem classified as NP is said to be NP-Complete.

\subsection{}
Since the left side of all production rules only contains single non-terminals it is said to be context-free.

At the same time \emph{G} is not regular (Type-3) due to multiple terminals and non-terminals on the right side of the production rules.

Therefore, the presented context-free non-regular grammar \emph{G} is of Type-2.

\section{Computational complexity theory}
\subsection{}
The 3-SAT problem is a boolean satisfiability problem. 

Given a formula in conjunctive normal form where each clause is limited to at most three literals, finding a configuration for the variables for which the formula evaluates to TRUE is NP-Complete. An \emph{instance} is for example:
\begin{equation*}
	F=(x_1 \lor x_2\lor x_3)\land(x_2\lor x_3 \lor x_4)\land(x_1 \lor x_2)
\end{equation*}

The \emph{decision} is to find a solution candidate in the search space which fullfills the given contraints.

\subsection{}
The vertex cover problem is a NP-Complete optimization problem in the graph theory. Given an \emph{instance} of a graph  $G=(V,E)$ the task (\emph{decision}) is to find a minimum set of vertices $V'$ for which hols:
\begin{equation*}
	\forall \: e \in E| e=(u,v):(u\in V')\lor (v \in V')
\end{equation*}

\subsection{}





\end{document}
